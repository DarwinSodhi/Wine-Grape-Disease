\documentclass[12pt,letter]{article}
\usepackage[top=1.00in, bottom=1.0in, left=1.1in, right=1.1in]{geometry}
\renewcommand{\baselinestretch}{1.1}
\usepackage{graphicx}
\usepackage{natbib}
\usepackage{amsmath}


\def\labelitemi{--}
\parindent=0pt

\begin{document}


\title{Winegrape Disease}
\author{Darwin Sodhi, Jonathan Davies & Elizabeth Wolkovich }
\date{October 2019}


\maketitle

\section{Introduction}

Pathogens play an important role in structuring communities (\cite{Janzen1970}, \cite{Connell1978}, \cite{Bush1997}) By definition, pathogens reduce the fitness of their hosts, and this in turn can influence how species coexists in a community (Mordecai.2011). For example, this reduction in species’ fitness is thought to stabilize communities by reducing dominance of one species (\cite{Chesson2000}). Pathogens are also though to promote diversity via the elevating seedling mortality closer to conspecific adult trees due to higher pathogen pressure (\cite{HilleRisLambers2012})---the Janzen-Connell hypothesis. The Janzen Connell hypothesis has been widely studied in the tropics (\cite{Clark2012}), but its importance in structuring temperate plant communities has been less well studied. 

\paragraph{}Plant pathogens can cause mortality, rapid declines of populations or large shifts in the structure of plant communities (\cite{Gilbert2002}). In agricultural settings, where there can be dense stands of conspecifics, pathogen outbreaks can be particularly devastating. Globally, about 16\% of all crops are lost to pathogens every year(\cite{Oerke2006}).Further, the yield loss due to pathogens can increase substantially if a pathogen is newly introduced to a new host (\cite{Oerke2006}).     
%[EMW: Feels like we need more of transition that sets up the problem statement here before you offer the solution.DSS: I added a few more sentences that hopefully make the transition smoother!]%
A better understanding of the ability for a pathogen to infect new hosts would allow for predictions on future disease pressure on communities (\cite{Borer2016}). Such predictions on future disease pressure can help in mitigating the potential of negative impacts of novel plant-pathogen interactions (\cite{Parker2015}). For example, the bacterium \textit{Xylella fastidiosa}, primarily regarded as a disease of olive trees (\textit{Olea europaea}), it was first described on trees in Italy, but has subsequently now been reported in north and south America, for which it has emerged as a threat to food production,timber production and other ecosystem services (\cite{Almeida2019}). Because \textit{Xylella fastidiosa} is known have a large host range consisting of over 560 species (EFSA.2012) and is known for its ability to easily jump to new hosts (\cite{Sicard2018}) the European Commission immediately implemented an eradication program; nonetheless, the disease still managed to significantly impact the agricultural output of several European countries (\cite{Almeida2019}). Knowing the host range and ability to host jump of \textit{Xylella fastidiosa} helped mitigate impacts by ensuring a timely and widespread response, without such knowledge there would have been a larger myriad of potential negative impacts (i.e: ecological, economic and social) for many of the infected areas.  The case study of \textit{Xylella fastidiosa} shows why it is imperative to understand what determines pathogen host breadth and the abilities of pathogens to jump from on host to another.

% EMW: The below is a very nice review, but I think you may need to highlight what is missing or needs more testing or better evidence better. Otherwise it almost reads as though this is all sorted and why would we need your work? Unless this is all sorted and you're just providing a test for winegrapes. We should make sure this is clear.
\paragraph{}Parker and Gilbert (2004\cite{Parker2015}) suggest that four factors are particularly critical in determining the probability of a host jump: (a) the degree of dependence of the pathogen on live hosts, (b) the degree of specialization of the pathogen, (c) the phylogenetic distance between the novel potential host and hosts with which the pathogen is familiar, and (d) the degree of ecological association between the pathogen and the potential host. The degree of dependence of the pathogen on live hosts can be related to the dispersal mode of the pathogen (\cite{Roy2001}). For example, many papers have shown that pathogens can still spread despite being grown on dead tissue, however this is still overlooked in many ecological studies on pathogen spread (\cite{Parker2015}). The specificity of a pathogen is a product of the range of hosts a pathogen infects (\cite{Parker2015}): highly specialized pathogens are less likely to be found on a wide range of hosts (Alberts.2002). Phylogenetic distance between new host and previous hosts has been shown to be a strong predictor of host shifts (\cite{Farr2002}, \cite{Leebens-Mack2006}, \cite{Duncan2002}). This has been hypothesized to be because phylogenies can capture ecologically relevant information on host physiology, immunology, and life history (\cite{Davies2008}, \cite{Gilbert2007}). However, unpredictable host-pathogen associations are known to occur despite strong phylogenetic patterns underlying most interactions (\cite{Biology1987}), proximity of hosts and pathogens, abiotic interactions needed for host shifts, and temporal overlap of hosts and pathogens (\cite{Parker2015}), can all mediate the outcome of potential host shifts. However predicting host jumping is notoriously difficult (\cite{Sicard2018}, \cite{Geoghegan2017}) and the need for it in agricultural settings is great (\cite{McDonald2016}).
% EMW: I wonder if winegrapes should be introduced here? They are introduced towards the end of the below paragraph, which is more focused on methods. I leave it to you and Jonathan to know the best approach, but to me I would think somewhere about here you want to get around to the aim of the work and then, sometime soon after, what group you'll be studying.
% DSS: I agree I think it flows nicer with the introduction to winegrapes here!

\paragraph{}Agricultural species provide a useful study system because agricultural plant pathogens are often well described. In addition, understanding threats posed by potential host shifts is critically important for food security (\cite{Bommarco2013}), and breeding for pathogen resistance is a top priority for cultivators. Winegrapes are a particularly useful model system because they are widely planted around the globe (\cite{IOVW}), they are susceptible to diverse pathogens differing in host breadth (\cite{Armijo2016}), and climate change is expected to result in susceptibility to a growing number of pathogens in the future (Orduna.2010). Winegrapes are also one of the most economically important crop species globally (\cite{IOVW}), and there has been much research on individual Winegrape diseases (\cite{Armijo2016}, \cite{Zahavi2000}, \cite{Hopkins2002}), providing a wealth of baseline data on important pathogens that affect Winegrape yield and health. 

\paragraph{}Phylogeny has been shown to be a good predictor of pathogen host range across various systems such as fungal pathogens of trees (\cite{Gilbert2007}), fish (\cite{Poulin1995}), and primates (\cite{Davies2008}). However, a diversity of phylogenetic metrics populate the literature, and metric choice requires careful consideration (\cite{Tucker2017}). Common metrics, such as PD (phylogenetic diversity), and ED (evolutionary distinctiveness) are frequently used in conservation biology (\cite{Isaac2007}), whereas pairwise distance metrics, such as MPD and MNTD, which are used to understand how phylogenetically related species are in a phylogeny, and how this relates to a null expectation (\cite{Kembel2010}) are more common in community ecology (Webb \textit{et al}.2008) and macroecology (\cite{DaviesJonathan2011}; \cite{Tucker2017}). More recently, related phylogenetic measures have been used to understand pathogen and host phylodynamics (\cite{Fountain-Jones2017}) and pathogen impacts on their hosts (\cite{Farrell2019}). Here, following Gilbert & Webb (2007 \cite{Gilbert2007}) and Davies & Pedersen (2008 \cite{Davies2008}), we use pairwise distance metrics to describe the host breadth of Winegrape pathogens. Limited knowledge on pathogen infections in wild species, especially a lack of data on non-infected hosts (\cite{Parker2015a}), has made it challenging to test predictions. Large databases and macro-ecological approaches have provided one way forward (see e.g. \cite{Stephens2016}). The more extensive data available for domesticated species provides an alternative approach (see e.g. \cite{Farrell2019}). Here we examine pathogens of Winegrape (\textit{Vitis vinifera}).


\paragraph{}In this paper we use three phylogenetic measurements, mean pairwise distance (MPD), mean nearest taxon distance (MPD) and focal distance to agricultural host species (Vitis vinifera), to capture the phylogenetic structure of 47 winegrape pests across over three thousand agricultural host species. We then describe the relationship between XXXXX and XXXX, and explore how pathogen traits (type of pathogen, body size, host specificity and reproductive rate) influence the phylogenetic structure of their distribution across hosts. If we can better predict emergence of winegrape pathogens, we can guide proactive surveillance of high-risk threats, and better prepare to mitigate their negative impacts. % ENW: Nice last sentence, we should make sure we build up to this so it's obvious to the reader by the time they read this!


\section{Methods}
\subsection{Data Collection}
We used three online resources (www.webofscience, www.cabi.org and www.scalenet.info) to build a preliminary list of the major pathogens affecting winegrapes and the list of agricultural host species that they also infect. We queried each database using the following search terms: wine grape pathogens, Vitis vinifera pathogens, and wine grape pathogen impacts, and recorded data for all species for which we could obtain published data on infection rates of \textit{Vitis vinifera} and impacts on yield. This returned a list of forty-nine (see supplementary list) pathogens, which we separated into four broad categories: fungal, bacterial, viral and pest. We cleaned taxonomic nomenclature, recorded all synonyms, and obtained a list of all hosts using information from http://www.dpvweb.net/,http://nemaplex.ucdavis.edu/, https://nt.ars-grin.gov/fungaldatabases/). The total host range for each pathogen was obtained by concatenating the list of unique hosts, ensuring that the final host list for each of our forty-nine pathogens included \textit{Vitis vinifera}. 
For each pathogen, we obtained georeferenced locations and native geographic distribution from cabi.org. Finally, conducted a literature search in Web Of Science (https://www.webofknowledge.com) identify how many papers had been published on each of our final set of 49 pathogens. We recorded the number of the total papers published, the number of papers including all synonyms, and number of papers in agricultural categories.  

\subsection{Host species and phylogeny}
To characterise the phylogenetic distribution of winegrape pathogens on their agricultural hosts, we first, obtained a list of agricultural host species (n = 944) from Milla et al. (2018). We then matched our host list to this subset, returning the list of agricultural hosts infected by each winegrape pathogen. Second, we subset the more inclusive and calibrated phylogenetic tree from Zanne et al. (2014\cite{Zanne2014}) to just the agricultural host species listed above. Because some hosts were only identified to genus, we performed two analyzes, one that assumed such pathogens infected all agricultural species in that genus, and another assuming only a single arbitrary host species within the genus was infected (ensuring that this species was represented in the Zanne et al.2014 phylogeny\cite{Zanne2014}). By assuming all host species in a genus are infected, we likely inflate taxonomic clustering of pathogen host range however, using only one single arbitrary specie would likely deflate taxonomic clustering of pathogen host range. By conducting the two analyses congruently we can better infer patterns by comparing and contrasting patterns found between the two analyses.  
\subsection{Phylogenetic metrics}
We used the PICANTE package in R to quantify the phylogenetic clustering of host species infected by each pathogen. Using the subset phylogeny from Zanne et al.(2014\cite{Zanne2014}), we calculated the mean pairwise distances (MPD) between all hosts for each pathogen, the mean nearest taxon distance (MNTD), which describes the minimum pairwise distance separating hosts for each pathogen, and the mean and minimum distance to \textit{Vitis vinifera} (Focal Distance). Standard effect sizes (SES) were calculated for each measure using the following equation, assuming a null model of no phylogenetic structure (option: tip swap in PICANTE), and 999 replicates. 
Standard Effect Size is Measured:  
\begin{align}
SES_{metric} & = \frac{Metric_{observed}-{mean({Metric_{null})}}}{SD({Metric_{null})}}
\end{align}

We explore variation in phylogenetic structure across pathogens using regression models and Bayesian analyses. First, we tested for broad differences across taxonomic groupings, separating pathogens into four categories: fungal, bacterial, viral and pest. Second, we evaluated differences between specialist (pathogens infecting only a single genus) and generalist (pathogens infecting multiple genera) pathogens. Finally, we tested for trait differences, using body size information for all nematodes and pests using websites: http://nemaplex.ucdavis.edu/ and cabi.org. Equivalent data for body size for bacteria and fungi are not available.All statistical analyses were carried out using the R statistical software (R Development Core Team, 2017).

\section{Results}
\subsection{MPD and MNTD}
Based on MPD and MNTD results, most pathogens are clustered on the host tree across both analyses (Figure 1). For MPD over 80\% of species show clustering, while for MNTD over 90\% of species show clustering. For 6 pathogens species, MPD and MNTD could not be calculated as they are only found on one agricultural host (\textit{Vitis vinifera}). 

\subsection{Patterns across analyzes}
In cases where host species had no discernable species name, clustering is stronger when we include all species in a genus compared to when we include only a single representative species. This result is expected, and likely reflects the of adding more species to the phylogenetic metric calculations. For the analysis including all the species in a genus the number of species added varied from 1-56 which represented a substantial increase in number of taxa for some pathogens. This patterns was most notable for MPD metric (Figure 2). 

\subsection{Contrasting patterns within analyzes}
There are cases where MPD and MNTD for a single pathogen (figure 3), show opposite trends. In two case we do observe overdispersion in MNTD and underdispersion in MPD, however in this case, the strenght of clustering is weak. An example of this is the pathogen \textit{Pseudococcus viburni} shows underdispersion for MPD (MPD.obs.z = -1.26) but weak overdispersion for MNTD (MNTD.obs.z = 0.0973). In more cases (4-6), we observe underdispersion in MNTD and overdispersion for MPD (Figure 3). An example of this is the pathogen \textit{Alternia alternata} shows overdispersion for MPD (MPD.obs.z = 0.501) but underdispersion for MNTD (MNTD.obs.z = -2.42).The threshold for significance is 1.96 for MNTD and MPD.

\subsection{Predicting host range}
Across both MPD and MNTD, our bayesian analyses show that pathogen type, and pathogen category are significant predictors of pathogen-host ranges.Further across analyses the patterns are consistent. In addition, specialist pathogens tend to have narrower host ranges to generalist pathogens. Body size was shown to be a poor predictor of host range in our analyzes, however only 15 out of the 49 pathogens were included in the analysis as we were only able to obtain body size measurements for these 15 pathogens. 

\subsection{Host jumping and impact}
Focal distances to Winegrape, most pathogens are clustered on the host tree. Over 80\% of species show clustering across both analyses. 

\section{Discussion}
 Generally, pathogens tend to infect host species that are phylogenetically similar to each other because these host species share traits that make them more susceptible to the same pathogens (\cite{Davies2008}, \cite{Poulin1995}). Conversely, more distantly related host species do not share as many traits, thus decreasing the chance that they will share pathogen species (Freeland 1983, \cite{Davies2008}). A pathogen's host range being phylogenetically clustered is a common result. Gilbert and colleagues in 2012\cite{Gilbert2012}, showed through the use of a large global database of pathogen records that host range is clustered for several different types of pathogens such as fungal, bacterial, viral, pests and nematodes. A pathogen's ability to infect a new host, which is closely related to an existing host could be explained by species with small evolutionary distances having similar traits (\cite{Pearse2009}). In an agricultural setting, this result points to farmers avoiding planting evolutionary or morphologically similar species together to avoid the potential of host jumping. Increased biodiversity has been shown to reduce disease risk (\cite{Pagan2012}, Baysal and Silme.2018, Altieri. 1999) however these patterns could be explained by amplification or dilution effects. Our results suggest that it is not just the diversity of hosts, but their phylogenetic relationships that is important,yet this facet of agricultural planning is often overlooked.  

\paragraph{}MPD and MNTD have been shown to show contrasting patterns between different communities (\cite{Mazel2016}). MPD and MNTD are used in tandem to show different processes operating at different parts of the phylogenetic tree (\cite{Kembel2010} and \cite{Mazel2016}).MPD captures patterns at deep phylogenetic depths, while MNTD captures patterns at shallow phylogenetic depths (swenson). Our results show that there are some cases where there is over dispersion in MPD and under dispersion in MNTD. This would suggest that in some cases, pathogen's host ranges have large evolutionary distances within the nodes (MPD) of host phylogeny while the opposite is true for the tips (MNTD) of the host phylogeny (\cite{Tsirogiannis2013}). We believe our results in which pathogens show opposite trends in MPD and MNTD may be in part explained by human-mediated movement and subsequent host jumps. Introductions of non-native hosts and pathogens can also result in the infection of a new host species by providing new opportunities for host jumping between pathogens and naïve hosts (e.g., Peeler et al. 2011) thus increasing the host ranges of pathogens. This would explain the patterns we observed in cases where MPD and MNTD show alternative trends. 

\paragraph{}Pathogens with large host ranges is one way in which a pathogen's chance of persistence can be increased. Several factors can explain a pathogen's ability to infect multiple host species (Pulliam & Dushoff 2009). However,the ability to predict phylogenetic host ranges is still poorly understood (Logdon et al.2014, Holmes.2013). A pathogen with a large host breathe, can be highly virulent in one host (a host it is most commonly found on) while exhibiting low virulence in another (a host it recently just jumped to). This have been shown in mammmals where evolutionary distance was shown to be a ood predictor of pathogen related mortality (\cite{Farrell2019}). The exact impact of each host will depend on how each host contributes to a pathogen's fitness (Regoes et al.2000, Gandon.2004, Rigaud et al.2010). Another cost of a large host breathe is the degree to which a pathogen can adapt to a host's defence against pathogens (Bowden & Drake.2013). If a pathogen only infects one host species, the pathogen should evolve to become highly proficient at dealing with the defences of that host. In multi-host pathogens, however, an adaptation in one host species may be unsuccessful in another host species (Elena et al.2009). Our analyses have shown that both the pathogen type and pathogen category can be used in order to predict the agricultural host ranges for Winegrape pathogens. This is particularly useful in the management of pathogen outbreaks as it will allow stakeholders to better grasp which pathogens carry the highest risk of spread, as pathogens with larger host ranges would be expected to be more successful in host jumping. In addition, the ability to predict pathogen host range can be important in the selection of the biological controls for invasive species (Barton.2004). An important next step in understanding pathogen dynamics would be the ability to predict host jumping and the virulence on a new host.

\paragraph{}###Need to rework this paragraph, Our results show that the larger the focal distance to winegrape, the larger the potential impact in terms of yield loss. Studies about host jumping have avoided measuring impacts altogether (Gilbert et al. 2015). Further our results are in contrast with those shown by Gilbert and colleagues in 2015, as they found a reduced impact of pathogens with a larger phylogenetic distance. Our results have profound impacts on how we manage pathogens, as it is clear reducing the abilities of pathogens to make large phylogenetic jumps should be prioritized. Many studies have shown that host jumping is less likely for phylogenetically divergent hosts (Futuyma et al., 1995; Reed & Hafner, 1997; Nishiguchi et al., 1998; Morehead & Feener, 2000; Gilbert & Webb, 2007; Refrégier et al., 2008). Although these host jumps may be infrequent in nature, however an agricultural setting would be presumed to offer an ample opportunity for pathogens to make large phylogenetic host jumps. Our results should be used in tandem with studies that test host jumping abilities of pathogens to better inform stakeholders on how to prevent future pathogens outbreaks. Our results also point to reducing the geographical distance that agricultural species are moved as this would increase the likelihood of novel host pathogen jumps.    


\bibliographystyle{unsrt}
\bibliography{Winegrape_paper.bib}

\end{document}
